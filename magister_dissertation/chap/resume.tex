%==================================================================================
%==================================================================================
% Document		:		Chapitre: Résumé
%
% Auteur		: 		Abdelkrime ARIES
% Encadreur		:		Dr. Omar NOUALI
% Co-encadreur	:		Mme. Houda OUFAIDA
% Établissement	:		ESI (Ecole Nationale Supérieure d'Informatique; ex. INI) 
% Adresse		:		Oued Smar, Alger, Algérie 
% Année			:		2012/2013
% Grade			:		Magister
% Discipline 	:		Informatique 
% Spécialité	:		IRM (Informatique Répartie et Mobile)
% Titre			:		Résumé automatique de textes
%
%==================================================================================
%==================================================================================

%==========================L'entete de chapitre====================================
%==================================================================================
 \ifx\wholebook\relax\else
  	\documentclass[a4paper,12pt,oneside]{../use/ESIthesis}
  	
  	\usepackage{amsmath,amssymb}             % AMS Math
\usepackage[utf8]{inputenc}
%\usepackage[T1]{fontenc} %,LAE 
\usepackage[T1]{fontenc}
%\usepackage[french,english]{babel}
\usepackage[frenchb]{babel}
\usepackage{microtype}

%\usepackage[left=2.5cm,right=2.5cm,top=2.5cm,bottom=2.5cm,includefoot,includehead,headheight=13.6pt]{geometry}
\usepackage[left=2.8cm,right=2.2cm,top=2.8cm,bottom=2.8cm,includefoot,includehead,headheight=13.6pt]{geometry}
%\usepackage[left=3.8cm,right=3.2cm,top=2.8cm,bottom=2.8cm,includefoot,includehead,headheight=13.6pt]{geometry}
%\usepackage[left=1.5in,right=1.3in,top=1.1in,bottom=1.1in,includefoot,includehead,headheight=13.6pt]{geometry}
\renewcommand{\baselinestretch}{1.5}

% Table of contents for each chapter

\usepackage[nottoc, notlof, notlot]{tocbibind}
\usepackage[french]{minitoc}
\setcounter{minitocdepth}{1}
\mtcindent=15pt
% Use \minitoc where to put a table of contents

\usepackage{aecompl}

% Glossary / list of abbreviations

\usepackage[intoc]{nomencl}
%\renewcommand{\nomname}{List of Abbreviations}

\makenomenclature

% My pdf code

\usepackage[pdftex]{graphicx}
\usepackage[a4paper,pagebackref,hyperindex=true]{hyperref}

%I added
%\usepackage{tabulary}
%\usepackage{longtable}
%\usepackage[table]{xcolor}
\usepackage{indentfirst}


% Links in pdf
\usepackage{color}
%\definecolor{linkcol}{rgb}{0,0,0.4} 
%\definecolor{citecol}{rgb}{0.5,0,0} 

% Change this to change the informations included in the pdf file

% See hyperref documentation for information on those parameters

\hypersetup
{
%bookmarksopen=true,
pdftitle=Résumé Automatique de Textes,
pdfauthor=Abdelkrime ARIES, 
pdfsubject= {Résumé automatique de textes en utilisant une approche statistique, le regroupement, et la classification} , %subject of the document
%%pdftoolbar=false, % toolbar hidden
%pdfmenubar=true, %menubar shown
%pdfhighlight=/O, %effect of clicking on a link
colorlinks=false, %couleurs sur les liens hypertextes
%pdfpagemode=None, %aucun mode de page
%pdfpagelayout=SinglePage, %ouverture en simple page
%pdffitwindow=true, %pages ouvertes entierement dans toute la fenetre
%linkcolor=linkcol, %couleur des liens hypertextes internes
%citecolor=citecol, %couleur des liens pour les citations
%urlcolor=linkcol %couleur des liens pour les url
}



% Some useful commands and shortcut for maths:  partial derivative and stuff

\newcommand{\pd}[2]{\frac{\partial #1}{\partial #2}}
\def\abs{\operatorname{abs}}
\def\argmax{\operatornamewithlimits{arg\,max}}
\def\argmin{\operatornamewithlimits{arg\,min}}
\def\diag{\operatorname{Diag}}
\newcommand{\eqRef}[1]{(\ref{#1})}

\usepackage{rotating}                    % Sideways of figures & tables
%\usepackage{bibunits}
%\usepackage[sectionbib]{chapterbib}          % Cross-reference package (Natural BiB)
%\usepackage{natbib}                  % Put References at the end of each chapter
                                         % Do not put 'sectionbib' option here.
                                         % Sectionbib option in 'natbib' will do.
\usepackage{fancyhdr}                    % Fancy Header and Footer

\usepackage{txfonts}                     % Public Times New Roman text & math font
  
%%% Fancy Header %%%%%%%%%%%%%%%%%%%%%%%%%%%%%%%%%%%%%%%%%%%%%%%%%%%%%%%%%%%%%%%%%%
% Fancy Header Style Options

\pagestyle{fancy}                       % Sets fancy header and footer
\fancyfoot{}                            % Delete current footer settings

%\renewcommand{\chaptermark}[1]{         % Lower Case Chapter marker style
%  \markboth{\chaptername\ \thechapter.\ #1}}{}} %

%\renewcommand{\sectionmark}[1]{         % Lower case Section marker style
%  \markright{\thesection.\ #1}}         %
%\fancyhead[LE,RO]{\bfseries\thepage}    % Page number (boldface) in left on even
%										% pages and right on odd pages
%\fancyhead[RE]{\bfseries\nouppercase{\leftmark}}      % Chapter in the right on even pages
%\fancyhead[LO]{\bfseries\nouppercase{\rightmark}}     % Section in the left on odd pages

\fancyhead[R]{\bfseries\thepage}    % Page number (boldface) in right
\fancyhead[L]{\bfseries\nouppercase{\rightmark}}     % Section in the left on odd pages

\let\headruleORIG\headrule
\renewcommand{\headrule}{\color{black} \headruleORIG}
\renewcommand{\headrulewidth}{1.0pt}
\usepackage{colortbl}
\arrayrulecolor{black}

\fancypagestyle{plain}{
  \fancyhead{}
  \fancyfoot{}
  \renewcommand{\headrulewidth}{0pt}
}

%\usepackage{algorithm}
%\usepackage[noend]{algorithmic}

%%% Clear Header %%%%%%%%%%%%%%%%%%%%%%%%%%%%%%%%%%%%%%%%%%%%%%%%%%%%%%%%%%%%%%%%%%
% Clear Header Style on the Last Empty Odd pages
\makeatletter

\def\cleardoublepage{\clearpage\if@twoside \ifodd\c@page\else%
  \hbox{}%
  \thispagestyle{empty}%              % Empty header styles
  \newpage%
  \if@twocolumn\hbox{}\newpage\fi\fi\fi}

\makeatother
 
%%%%%%%%%%%%%%%%%%%%%%%%%%%%%%%%%%%%%%%%%%%%%%%%%%%%%%%%%%%%%%%%%%%%%%%%%%%%%%% 
% Prints your review date and 'Draft Version' (From Josullvn, CS, CMU)
\newcommand{\reviewtimetoday}[2]{\special{!userdict begin
    /bop-hook{gsave 20 710 translate 45 rotate 0.8 setgray
      /Times-Roman findfont 12 scalefont setfont 0 0   moveto (#1) show
      0 -12 moveto (#2) show grestore}def end}}
% You can turn on or off this option.
% \reviewtimetoday{\today}{Draft Version}
%%%%%%%%%%%%%%%%%%%%%%%%%%%%%%%%%%%%%%%%%%%%%%%%%%%%%%%%%%%%%%%%%%%%%%%%%%%%%%% 

\newenvironment{maxime}[1]
{
\vspace*{0cm}
\hfill
\begin{minipage}{0.5\textwidth}%
%\rule[0.5ex]{\textwidth}{0.1mm}\\%
\hrulefill $\:$ {\bf #1}\\
%\vspace*{-0.25cm}
\it 
}%
{%

\hrulefill
\vspace*{0.5cm}%
\end{minipage}
}

\let\minitocORIG\minitoc
\renewcommand{\minitoc}{\minitocORIG \vspace{1.5em}} %1.5em

\usepackage{multirow}
%\usepackage{slashbox}

\newenvironment{bulletList}%
{ \begin{list}%
	{$\bullet$}%
	{\setlength{\labelwidth}{25pt}%
	 \setlength{\leftmargin}{30pt}%
	 \setlength{\itemsep}{\parsep}}}%
{ \end{list} }

\newtheorem{definition}{Définition }
\renewcommand{\epsilon}{\varepsilon}

% centered page environment

\newenvironment{vcenterpage}
{\newpage\vspace*{\fill}\thispagestyle{empty}\renewcommand{\headrulewidth}{0pt}}
{\vspace*{\fill}}

%%%%%%%%%%%%%%%%%%%%%%%%%%%%%%%%%%%%%%%%%%%%%%%%%%%%%%%%%%%%%%%%%%%%
% Par Karim
%%%%%%%%%%%%%%%%%%%%%%%%%%%%%%%%%%%%%%%%%%%%%%%%%%%%%%%%%%%%%%%%%%%%
%for the degree sign
\usepackage{textcomp} 
\usepackage{bookmark}
\usepackage{framed}
\usepackage{arabtex}
%\usepackage{nashbf}
%\usepackage{atrans}
%calligra font for the remerciement
\usepackage{calligra}

%List of acronyms
\usepackage{acronym}

\newcommand{\racine}{./}

\newcommand{\setracine}[1]{\renewcommand{\racine}{#1}}

\newcommand{\tablefile}[1]{\input{\racine tab/#1}}
\newcommand{\appendixfile}[1]{\input{\racine anx/#1}}
%\newcommand{\chapterfile}[1]{\input{\racine chap/#1}}

\newcommand{\stitle}[1]{
\noindent
\textbf{#1}
}

\newenvironment{itemizeb}
{\begin{list}{\textbullet} {\setlength{\rightmargin}{0cm} \setlength{\leftmargin}{1cm}}}
{\end{list}}


\newenvironment{itemizec}
{\begin{list}{\textopenbullet} {\setlength{\rightmargin}{0cm} \setlength{\leftmargin}{1cm}}}
{\end{list}}


\newcommand{\kexpbox}[1]{

\vspace{5mm}
\noindent
 \fbox{%
   \parbox{0.985\linewidth}{%
   \vspace{2mm}
   {\large  \textbf{Exemple:}}\\
      #1
   }%
 }
}

\newcommand{\kbox}[1]{

\vspace{2mm}
\noindent
 \fbox{%
   \parbox{0.965\linewidth}{%
   \vspace{2mm}
      #1
   }%
 }
}

\newenvironment{kexp}
{
\begin{framed}
\noindent
{\large  \textbf{Exemple:}}\\
}
{
\end{framed}
}

%%%%%%%%%%%%%%%%%%%%%%%%%%%%%%%%%%%%%%%%%%%%%%%%%%%%%%%%%%%%%%%%%%%%
%%%%%%%%%%%%%%%%%%%%%%%%%%%%%%%%%%%%%%%%%%%%%%%%%%%%%%%%%%%%%%%%%%%%

% definitions.
% -------------------

\setcounter{secnumdepth}{3}
\setcounter{tocdepth}{2}

\newcommand{\tab}[1]{{\hskip #1}}
  	 	
  	 	\setracine{../}
  	 	\graphicspath{{.}{../fig/}}
  	 	
  	 	\begin{document}
  	 	
  	 	%\dominitoc 
  	 	\selectlanguage {francais}
  	 	%just to create the .toc file, then you can hide it
  	 	%\tableofcontents
  	 	\frontmatter
  \fi
%==================================================================================

\chapter*{Résumé}

Dans le cadre de notre mémoire, nous avons travaillé sur le résumé automatique de texte. 
Notre objectif était de développer un système général; qui est indépendant de la langue et du genre du texte d'entrée. 
Notre choix s'est dirigé vers l'utilisation d'une approche statistique (numérique) à cause de sa dépendance faible au langue par rapport à l'approche linguistique. 
Dans l'approche statistique, le système utilise un groupe de caractéristiques pour juger la pertinence des unités à extraire (phrases en général). 

Pour combiner les scores des différentes caractéristiques dans un seul score, nous avons utilisé un algorithme d'apprentissage. 
Contrairement aux systèmes basant sur l'apprentissage, notre système n'utilise pas un corpus d'entrainement, puisque ce dernier est dépendant à certaine langue et certain genre. 
Notre idée est de choisir les phrases qui peuvent représenter le texte d'entrée au totalité. 
Puisque le texte d'entrée se compose de plusieurs thèmes, une phrase du résumé est supposée représenter tous ou la majorité de ses thèmes, prenant en considération l'importance de chaque thème. 

Nous avons utilisé un algorithme de regroupement pour regrouper les phrases similaires dans des clusters. 
Ensuite, nous avons entrainé notre système sur ces différents clusters pour avoir des modèles. 
Pour chaque phrase, nous avons calculé la probabilité qu'elle peut représenter chacun de ces clusters, étant donné un vecteur de caractéristiques. 
Enfin, ces probabilités sont combinées dans un score, ce dernier est utilisé pour réordonner les phrases pour extraire celles plus pertinentes.

\vspace*{\fill}

\textbf{Mots-clés: }
Résumé automatique, 
extraction, 
Regroupement, 
Classification, 
Théorie de Bayes
.

\clearpage

\chapter*{Summary}

As a part of our Magister thesis, we work on automatic text summarization. 
Our objective was to develop a general system, which is input text's language and genre independent. 
As a choice, we use statistical (numerical) approach which is less language dependent then the linguistic approach. 
In the statistical approach, the system use a set of features to calculate the relevancy of extracted units (sentences in general). 

To combine the scores of different features in one score, we used a learning algorithm. 
Unlike the learning-based systems, ours don't use a training corpus, which is language and genre dependent. 
Our idea is to select the sentences which can represent the whole input text. 
Because the input text is composed of many topics, a sentence of the summary is supposed to represent all or the majority of these topics, considering each topic's importance. 

We used a clustering algorithm to create similar sentences' clusters. 
Then, We trained our system on these clusters to find classification models. 
For each sentence, we calculated the probability that it can represent each of these clusters, knowing a features vector.
Finally, these probabilities are combined in one score, which can used to reorganize the sentences and extract the more relevant ones.

\vspace*{\fill}


\textbf{Keywords: }
Automatic summarization, 
extraction, 
Clustering, 
Classification, 
Bayes theory
.

\clearpage

\chapter*{\RL{mulaxxa.s}}

\begin{arabtext}
fI 'i.tAri mu_dakkiratinA ha_diHi, qumnA bi-al`amali `alY at-talxI.s al-tilqA'iy li-l-nno.sO.s.
hadafunA huwa ta.twIr ni.zAm `Am, mustaqill `ani al-lu.gaT wa-al-'uslUb. 
li_dA ixtarnA isti`mAl al-manha^g al-'i.h.sA'iy (ar-raqmy) li-'annahu nAdiraN mA ya`tamidu `lY al-ma`rifaT al-lu.gawiyyaT muqAranaTaN ma`a al-manha^g al-lu.gawy. 
fy al-manha^g al-'i.h.sA'iy, yasta`milu an-ni.zAm ma^gmU`aTaN mina al-myzAt min 'a^gli taqdyri al-mu.htawY ad-dalAly li-l-wa.hadAt al-mustaxla.saT (^gumal fy 'a.glabi al-'a.hyAn).

li-ta^gmy`i an-natA'iji al-muxtalifaT li-l-myzAt fy natyjaTiN wA.hidaTiN, ista`malnA tiqniyyaT at-ta`allum al-'Aly. 
bi-`aksi al-'ab.hA_ti as-sAbiqaTi al-murtakizaTi `alY at-ta`allum al-'Aly, ni.zAmunA lA yasta`mil 'ayyaTa _daxA'ir nu.sU.s li-l-ttadryb, li-'anna ha_dihi al-'axyraT murtabi.taTuN bi-lu.gaTiN wa-'uslUbiN mu`ayyanayni.
fikratunA hiya 'an naqUma bi-axtiyAri al^gumali allaty tasta.ty`u at-ta`byra `ani an-na.s.si 'i^gmAlaN. 
ma`a `ilminA 'anna an-na.sa yatakawwanu min `iddaTi mawA.dy`, fa-'inna al-mulaxxa.sa ya.htawy `lY al-jumali allaty tasta.ty`u at-ta`byra `an kulli 'aw mu`.zam ha_dihi al-mawA.dy`, 'Axi_dyna bi`ayni al-i`tibAr 'ahammiyyaTa kulli maw.dU`.

'ista`malnA xawArizm xA.s bi-at-ta^gmy` min 'a^gli 'in^sA' ma^gmU`At ta.htawy `alY al-^gumal al-muta^sAbihaTi. 
ba`dahA, qumnA bi-tadryb ni.zAminA `alY muxtalafi al-ma^gmU`At li-l-.hu.sUli `lY namA_di^g lit-ta.snIf. 
_tumma qumnA bi-.hisAb i.htimAl 'an tu`abbira ^gumlaTuN mA `alY kulli ma^gmU`aTiN, 'Ax_dyna bi-`ayni al-i`tibAr ba`.d al-myzAt al-mu.haddadaT. 
'axyraN, ha_dihi al-i.htimAlAt ^gummi`at fy naty^gaTiN wA.hidaTiN tusta`malu li-tartyb al-^gumal wa-istixlA.s al-'ak_tar 'ahammiyyaTaN minhA.
\end{arabtext}

\vspace*{\fill}

\begin{arabtext}

{\bf alkalimAt almiftA.hyyaT -} at-talxI.s at-tilqA'iy, al-istixlA.s, at-ta^gmI`, at-ta.snIf, na.zaryyaTu bAyz.

%extraction, 
%Clustering, 
%Classification, 
%Na\"ive Bayes
\end{arabtext}

\clearpage

\ifx\wholebook\relax\else
\cleardoublepage
    \end{document}
\fi