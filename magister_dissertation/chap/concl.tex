%==================================================================================
%==================================================================================
% Document		:		Conclusion
%
% Auteur		: 		Abdelkrime ARIES
% Encadreur		:		Dr. Omar NOUALI
% Co-encadreur	:		Mme. Houda OUFAIDA
% Établissement	:		ESI (Ecole Nationale Supérieure d'Informatique; ex. INI) 
% Adresse		:		Oued Smar, Alger, Algérie 
% Année			:		2012/2013
% Grade			:		Magister
% Discipline 	:		Informatique 
% Spécialité	:		IRM (Informatique Répartie et Mobile)
% Titre			:		Résumé automatique de textes
%
%==================================================================================
%==================================================================================

%==========================L'entete de chapitre====================================
%==================================================================================
 \ifx\wholebook\relax\else
  	\documentclass[a4paper,12pt,oneside]{../use/ESIthesis}
  	
  	\usepackage{amsmath,amssymb}             % AMS Math
\usepackage[utf8]{inputenc}
%\usepackage[T1]{fontenc} %,LAE 
\usepackage[T1]{fontenc}
%\usepackage[french,english]{babel}
\usepackage[frenchb]{babel}
\usepackage{microtype}

%\usepackage[left=2.5cm,right=2.5cm,top=2.5cm,bottom=2.5cm,includefoot,includehead,headheight=13.6pt]{geometry}
\usepackage[left=2.8cm,right=2.2cm,top=2.8cm,bottom=2.8cm,includefoot,includehead,headheight=13.6pt]{geometry}
%\usepackage[left=3.8cm,right=3.2cm,top=2.8cm,bottom=2.8cm,includefoot,includehead,headheight=13.6pt]{geometry}
%\usepackage[left=1.5in,right=1.3in,top=1.1in,bottom=1.1in,includefoot,includehead,headheight=13.6pt]{geometry}
\renewcommand{\baselinestretch}{1.5}

% Table of contents for each chapter

\usepackage[nottoc, notlof, notlot]{tocbibind}
\usepackage[french]{minitoc}
\setcounter{minitocdepth}{1}
\mtcindent=15pt
% Use \minitoc where to put a table of contents

\usepackage{aecompl}

% Glossary / list of abbreviations

\usepackage[intoc]{nomencl}
%\renewcommand{\nomname}{List of Abbreviations}

\makenomenclature

% My pdf code

\usepackage[pdftex]{graphicx}
\usepackage[a4paper,pagebackref,hyperindex=true]{hyperref}

%I added
%\usepackage{tabulary}
%\usepackage{longtable}
%\usepackage[table]{xcolor}
\usepackage{indentfirst}


% Links in pdf
\usepackage{color}
%\definecolor{linkcol}{rgb}{0,0,0.4} 
%\definecolor{citecol}{rgb}{0.5,0,0} 

% Change this to change the informations included in the pdf file

% See hyperref documentation for information on those parameters

\hypersetup
{
%bookmarksopen=true,
pdftitle=Résumé Automatique de Textes,
pdfauthor=Abdelkrime ARIES, 
pdfsubject= {Résumé automatique de textes en utilisant une approche statistique, le regroupement, et la classification} , %subject of the document
%%pdftoolbar=false, % toolbar hidden
%pdfmenubar=true, %menubar shown
%pdfhighlight=/O, %effect of clicking on a link
colorlinks=false, %couleurs sur les liens hypertextes
%pdfpagemode=None, %aucun mode de page
%pdfpagelayout=SinglePage, %ouverture en simple page
%pdffitwindow=true, %pages ouvertes entierement dans toute la fenetre
%linkcolor=linkcol, %couleur des liens hypertextes internes
%citecolor=citecol, %couleur des liens pour les citations
%urlcolor=linkcol %couleur des liens pour les url
}



% Some useful commands and shortcut for maths:  partial derivative and stuff

\newcommand{\pd}[2]{\frac{\partial #1}{\partial #2}}
\def\abs{\operatorname{abs}}
\def\argmax{\operatornamewithlimits{arg\,max}}
\def\argmin{\operatornamewithlimits{arg\,min}}
\def\diag{\operatorname{Diag}}
\newcommand{\eqRef}[1]{(\ref{#1})}

\usepackage{rotating}                    % Sideways of figures & tables
%\usepackage{bibunits}
%\usepackage[sectionbib]{chapterbib}          % Cross-reference package (Natural BiB)
%\usepackage{natbib}                  % Put References at the end of each chapter
                                         % Do not put 'sectionbib' option here.
                                         % Sectionbib option in 'natbib' will do.
\usepackage{fancyhdr}                    % Fancy Header and Footer

\usepackage{txfonts}                     % Public Times New Roman text & math font
  
%%% Fancy Header %%%%%%%%%%%%%%%%%%%%%%%%%%%%%%%%%%%%%%%%%%%%%%%%%%%%%%%%%%%%%%%%%%
% Fancy Header Style Options

\pagestyle{fancy}                       % Sets fancy header and footer
\fancyfoot{}                            % Delete current footer settings

%\renewcommand{\chaptermark}[1]{         % Lower Case Chapter marker style
%  \markboth{\chaptername\ \thechapter.\ #1}}{}} %

%\renewcommand{\sectionmark}[1]{         % Lower case Section marker style
%  \markright{\thesection.\ #1}}         %
%\fancyhead[LE,RO]{\bfseries\thepage}    % Page number (boldface) in left on even
%										% pages and right on odd pages
%\fancyhead[RE]{\bfseries\nouppercase{\leftmark}}      % Chapter in the right on even pages
%\fancyhead[LO]{\bfseries\nouppercase{\rightmark}}     % Section in the left on odd pages

\fancyhead[R]{\bfseries\thepage}    % Page number (boldface) in right
\fancyhead[L]{\bfseries\nouppercase{\rightmark}}     % Section in the left on odd pages

\let\headruleORIG\headrule
\renewcommand{\headrule}{\color{black} \headruleORIG}
\renewcommand{\headrulewidth}{1.0pt}
\usepackage{colortbl}
\arrayrulecolor{black}

\fancypagestyle{plain}{
  \fancyhead{}
  \fancyfoot{}
  \renewcommand{\headrulewidth}{0pt}
}

%\usepackage{algorithm}
%\usepackage[noend]{algorithmic}

%%% Clear Header %%%%%%%%%%%%%%%%%%%%%%%%%%%%%%%%%%%%%%%%%%%%%%%%%%%%%%%%%%%%%%%%%%
% Clear Header Style on the Last Empty Odd pages
\makeatletter

\def\cleardoublepage{\clearpage\if@twoside \ifodd\c@page\else%
  \hbox{}%
  \thispagestyle{empty}%              % Empty header styles
  \newpage%
  \if@twocolumn\hbox{}\newpage\fi\fi\fi}

\makeatother
 
%%%%%%%%%%%%%%%%%%%%%%%%%%%%%%%%%%%%%%%%%%%%%%%%%%%%%%%%%%%%%%%%%%%%%%%%%%%%%%% 
% Prints your review date and 'Draft Version' (From Josullvn, CS, CMU)
\newcommand{\reviewtimetoday}[2]{\special{!userdict begin
    /bop-hook{gsave 20 710 translate 45 rotate 0.8 setgray
      /Times-Roman findfont 12 scalefont setfont 0 0   moveto (#1) show
      0 -12 moveto (#2) show grestore}def end}}
% You can turn on or off this option.
% \reviewtimetoday{\today}{Draft Version}
%%%%%%%%%%%%%%%%%%%%%%%%%%%%%%%%%%%%%%%%%%%%%%%%%%%%%%%%%%%%%%%%%%%%%%%%%%%%%%% 

\newenvironment{maxime}[1]
{
\vspace*{0cm}
\hfill
\begin{minipage}{0.5\textwidth}%
%\rule[0.5ex]{\textwidth}{0.1mm}\\%
\hrulefill $\:$ {\bf #1}\\
%\vspace*{-0.25cm}
\it 
}%
{%

\hrulefill
\vspace*{0.5cm}%
\end{minipage}
}

\let\minitocORIG\minitoc
\renewcommand{\minitoc}{\minitocORIG \vspace{1.5em}} %1.5em

\usepackage{multirow}
%\usepackage{slashbox}

\newenvironment{bulletList}%
{ \begin{list}%
	{$\bullet$}%
	{\setlength{\labelwidth}{25pt}%
	 \setlength{\leftmargin}{30pt}%
	 \setlength{\itemsep}{\parsep}}}%
{ \end{list} }

\newtheorem{definition}{Définition }
\renewcommand{\epsilon}{\varepsilon}

% centered page environment

\newenvironment{vcenterpage}
{\newpage\vspace*{\fill}\thispagestyle{empty}\renewcommand{\headrulewidth}{0pt}}
{\vspace*{\fill}}

%%%%%%%%%%%%%%%%%%%%%%%%%%%%%%%%%%%%%%%%%%%%%%%%%%%%%%%%%%%%%%%%%%%%
% Par Karim
%%%%%%%%%%%%%%%%%%%%%%%%%%%%%%%%%%%%%%%%%%%%%%%%%%%%%%%%%%%%%%%%%%%%
%for the degree sign
\usepackage{textcomp} 
\usepackage{bookmark}
\usepackage{framed}
\usepackage{arabtex}
%\usepackage{nashbf}
%\usepackage{atrans}
%calligra font for the remerciement
\usepackage{calligra}

%List of acronyms
\usepackage{acronym}

\newcommand{\racine}{./}

\newcommand{\setracine}[1]{\renewcommand{\racine}{#1}}

\newcommand{\tablefile}[1]{\input{\racine tab/#1}}
\newcommand{\appendixfile}[1]{\input{\racine anx/#1}}
%\newcommand{\chapterfile}[1]{\input{\racine chap/#1}}

\newcommand{\stitle}[1]{
\noindent
\textbf{#1}
}

\newenvironment{itemizeb}
{\begin{list}{\textbullet} {\setlength{\rightmargin}{0cm} \setlength{\leftmargin}{1cm}}}
{\end{list}}


\newenvironment{itemizec}
{\begin{list}{\textopenbullet} {\setlength{\rightmargin}{0cm} \setlength{\leftmargin}{1cm}}}
{\end{list}}


\newcommand{\kexpbox}[1]{

\vspace{5mm}
\noindent
 \fbox{%
   \parbox{0.985\linewidth}{%
   \vspace{2mm}
   {\large  \textbf{Exemple:}}\\
      #1
   }%
 }
}

\newcommand{\kbox}[1]{

\vspace{2mm}
\noindent
 \fbox{%
   \parbox{0.965\linewidth}{%
   \vspace{2mm}
      #1
   }%
 }
}

\newenvironment{kexp}
{
\begin{framed}
\noindent
{\large  \textbf{Exemple:}}\\
}
{
\end{framed}
}

%%%%%%%%%%%%%%%%%%%%%%%%%%%%%%%%%%%%%%%%%%%%%%%%%%%%%%%%%%%%%%%%%%%%
%%%%%%%%%%%%%%%%%%%%%%%%%%%%%%%%%%%%%%%%%%%%%%%%%%%%%%%%%%%%%%%%%%%%

% definitions.
% -------------------

\setcounter{secnumdepth}{3}
\setcounter{tocdepth}{2}

\newcommand{\tab}[1]{{\hskip #1}}
  	 	
  	 	\setracine{../}
  	 	\graphicspath{{.}{../fig/}}
  	 	
  	 	\begin{document}
  	 	
  	 	%\dominitoc 
  	 	\selectlanguage {francais}
  	 	%just to create the .toc file, then you can hide it
  	 	%\tableofcontents
  	 	\mainmatter
  \fi
%==================================================================================

\chapter*{Conclusion et perspectives}
\label{chap:concl}
\addcontentsline{toc}{chapter}{Conclusion et perspectives}
\fancyhead[L]{\bfseries\nouppercase{Conclusion et perspectives}}

%Pour conclure ce mémoire, nous récapitulerons d'abord l'essentiel du travail réalisé, en insistant sur les points importants qui particularisent notre approche par rapport aux autres travaux. 
%Ensuite, nous discuterons les différents testes et les résultats obtenues.
%Enfin, nous terminerons en exposant les principales perspectives qui nous semblent les plus bénéfiques pour continuer à assurer l'intérêt et la solidité de notre méthode.

\section*{Travail réalisé}

Dans ce travail notre objectif était de développer une méthode de résumé automatique la plus générale possible (indépendante du genre et de la langue). 
%Comme nous avons présenté dans l'état de l'art, il existe deux approches de résumé automatique: l'approche statistique et l'approche linguistique. 
Sachant qu'il existe deux grandes approches de résumé automatique: l'approche statistique et l'approche linguistique,  l'utilisation de cette dernière rend le système dépendant à une certaine langue, et même à un certain genre.
Ainsi, notre choix s'est penché vers le développement d'une approche statistique (ou numérique), serte limitée à l'extraction, mais qui a prouvé son utilité dans le domaine. 
Elle utilise des critères statistiques pour juger de la pertinence d'une phrase (ou une autre unité à extraire). 
Pour comparer les phrases et surtout les ordonner, nous avons défini un score qui est la combinaison des critères multiplié par leurs poids. 
Ces poids sont, souvent, obtenus en utilisant l'apprentissage basé sur un corpus d'entraînement. 
Ceci rend le système dépendant de la langue et du genre du corpus d'entraînement, d'où l'idée d'utiliser l'apprentissage différemment. 
En effet, à partir du texte d'entrée, on identifie les thèmes abordés.
Chaque thème contient un ensemble de phrases similaires.
Pour ce faire, nous avons proposé un algorithme de regroupement basé sur leurs similarités (dans notre cas: similarité Cosinus). 
Ensuite, pour chaque thème, on extrait un modèle en utilisant un algorithme d'apprentissage, Na\"ive Bayes dans notre cas, et un ensemble de critères statistiques. 
Ces modèles sont utilisés pour calculer la probabilité qu'une phrase représente les différents thèmes du texte d'entrée. 
Enfin, les phrases sont triées selon leurs scores, afin d'en extraire les premières phrases. 

%Notre méthode utilise le regroupement pour identifier les différents thèmes dans le texte d'entrée, de la même façon que les travaux précédents. 
La différence entre notre travail et les travaux précédents se résume dans les points suivants: 
\begin{itemize}
\item Les autres méthodes avec regroupement, tentent de sélectionner une phrase représentante de chaque cluster. 
Au contraire, dans notre méthode, on tente de sélectionner les phrases qui représentent le maximum de clusters. 

\item Les méthodes par apprentissage, utilisent les algorithmes de classification pour trouver les poids des critères, ou bien pour décider si une phrase appartient au résumé ou non. 
Dans notre méthode, l'algorithme de classification est utilisé comme un moyen de notation. 

\item Notre méthode n'utilise pas un corpus d'entraînement, elle est ainsi indépendante du genre et de la langue du document source.
\end{itemize}

\section*{Les résultats obtenus}

Afin d'évaluer notre système dans le contexte du résumé mono-document, nous avons conduit des expérimentation en utilisant le corpus cmp-lg. 
%Ces expérimentations sont destinées pour accomplir les trois buts suivants: 
%\begin{itemize}
%\item La comparaison de notre système avec un notre (UNIS dans notre cas), en terme de qualité. 
%\item Tester l'effet d'ajouter un nouveau critère au système. 
%\item Adaptation de seuil de regroupement pour voir son effet sur le résumé résultant.
%\end{itemize}
%Les résultats ont montré que notre système surpasse UNIS dans la plupart des cas. 
Les résultats ont montré que notre système donne de bonnes performances. 
L'ajout de critères améliore la précision du système. 
%Pour le seuil de regroupement, on peut voir qu'il a un grand effet sur le résumé résultat. 

Motivé par ces résultats, nous avons appliqué notre méthode dans le contexte du résumé multi-documents. 
Pour ce faire, nous avons proposé deux stratégies de regroupement: la première associe un cluster à chaque document, la deuxième fusionne tous les documents d'entrée et regroupe les phrases similaires dans un même cluster. 
Afin d'évaluer les deux stratégies, nous avons conduit des expérimentations en utilisant le corpus DUC 2004. 
%\begin{itemize}
%\item Comparer les deux méthodes décrites pour le résumé multi-document.
%\item Positionner notre système par rapport aux autres systèmes participant dans DUC 2004.
%\item Évaluer l'impact de l'ajout de nouveaux critères sur la performance de notre système pour le résumé multi-documents. 
%\item Évaluer l'apport du regroupement pour notre système pour le résumé multi-document.
%\item Évaluer l'impact de normalisation sur le rendement de système.
%\item Évaluer l'impact de réduction de phrases sur le rendement de système.
%\end{itemize}
Nous avons constaté que la deuxième stratégie donne de meilleurs résultats.
Ceci peut être dû au fait que chaque document contient un thème principal et des thèmes secondaires, et en fusionnant les documents on a pu effectuer l'apprentissage sur chaque thème à part. 
%En comparant notre système avec les participants de DUC 2004, on peut dire que notre système donne des résultats satisfaisants. 
L'ajout de critères reste bénéfique et permet aussi d'améliorer le rappel et la précision du système.
%et du regroupement, on peut voir les mêmes observations que celles du résumé mono-document. 
%Nous avons utilisé la normalisation afin de ne pas favoriser les phrases longues sur celles courtes (ou réduite). 
L'ajout d'un module de compression a permis d'améliorer, dans certains cas, les performances.
%Ensuite, nous avons utilisé un module de compression de phrases très simple, qui nous a donnée des résultats encourageantes pour continue dans ce chemin. 

\section*{Perspectives et travaux à venir}

Nous avons testé notre système en utilisant deux critères: la fréquence des uni-grammes et des bi-grammes, les résultats étaient satisfaisants. 
Dans le futur, on compte évaluer l'effet engendré par d'autres critères: la longueur de phrase, sa position, etc., sur notre système. 
Motivés par la recherche de \cite{10-yatsko-al}, on compte aussi explorer la pertinence de ces critères par rapport à un genre de documents particulier. 
De plus, il est important d'identifier les meilleurs valeurs du seuil de regroupement, ceci dans le but d'effectuer un regroupement optimisé ni trop strict ni trop vague.
Enfin, du fait que notre méthode soit d'ordre général, des expériences devraient être menées sur d'autres domaines (articles de presse, romans, etc.) et d'autres langues (arabe, français, etc.).


%========================Le pied de chapitre=======================================
%==================================================================================
\ifx\wholebook\relax\else
 \cleardoublepage
 \bibliographystyle{../use/ESIbib} 
 \bibliography{../bib/evalMine}
 \end{document}
\fi
%===================================================================================
